\section{Introduction}

Dragonfly will investigate prebiotic chemistry and habitability on Saturn's largest moon, Titan, in the mid-2030s.\cite{barnes2021science,lorenzTitan} Successful exploration of Titan will require a series of multi-kilometer surface flights to access materials in locations with diverse geological histories.\cite{lorenz2018dragonfly, mcgee2018guidance} This paper highlights the preliminary design of the navigation (nav) filter for the Mobility subsystem, which is responsible for the guidance, navigation, and control (GNC) of the relocatable rotorcraft. In addition to operating autonomously, navigating on Titan presents several unique challenges. First, no external infrastructure or orbital assets are available for navigation. Surface maps constructed from Cassini flybys are too coarse for navigating near the surface, and the presence of diverse terrain and unknown topography require robust algorithms with limited a priori knowledge.~\cite{Lorenz_2021} Further, there are no external attitude references (e.g. stars) near the Titan surface, nor does a useful magnetic field exist, precluding the use of external sensors for attitude knowledge. 

Since high resolution maps of the Titan surface are not available, Dragonfly leverages \ac{VIO}, where an \ac{IMU} and sequential images are used to estimate how far the vehicle has traveled. \ac{VIO} has been successfully flown on NASA's MER-DIMES and Ingenuity missions.\cite{MER, ingenuityNav} One limitation of \ac{VIO} is the lack of observability of the absolute vehicle position, causing position errors to accumulate over time. On Dragonfly the impact of accumulated error is mitigated by periodically saving an image for future navigation.\cite{mcgee2018guidance}  The saved images, or "breadcrumbs", are used to navigate back to the take-off site, or to a pre-scouted landing site using the leapfrog approach.\cite{mcgee2018guidance, witte2019no, schilling2019} When revisiting breadcrumbs, the nav filter must fuse correlated position measurements, a variation of the \ac{SLAM} problem.\cite{smith1990estimating, moutarlier1991incremental} The traditional \ac{SLAM} formulation of maintaining many landmarks (i.e. breadcrumbs) in the filter state and updating landmarks during revisits is intractable on the flight processor selected for Dragonfly, serving as motivation for an efficient and robust approximation to the \ac{SLAM} formulation.

This paper focuses on design challenges related to long distance traverse navigation on Titan. The \ac{EDL} sequence offers different challenges for navigation including large initial attitude errors with respect to gravity and high angular rates throughout the parachute descent phase. Navigation for \ac{EDL} is achieved using the same nav filter described in this paper and will be the topic of a future paper. The organization of this paper follows as: the~\nameref{background} section discusses material relevant to the Dragonfly mission and nav filter design; the~\nameref{design} section outlines the preliminary design, including the architecture, state space, and measurement models; the~\nameref{results} section illustrates performance from a high fidelity Titan simulation, and finally the~\nameref{conclusion} section summarizes results and highlights future work.