\begin{abstract}
The mobile robot relies on SLAM (Simultaneous Localization and Mapping) to provide autonomous navigation and task execution in complex and unknown environments. However, it is hard to develop a dedicated algorithm for mobile robots due to dynamic and challenging situations, such as poor lighting conditions and motion blur. To tackle this issue, we propose a tightly-coupled LiDAR-visual SLAM based on geometric features, which includes two sub-systems (LiDAR and monocular visual SLAM) and a fusion framework. The fusion framework associates the depth and semantics of the multi-modal geometric features to complement the visual line landmarks and to add direction optimization in Bundle Adjustment (BA). This further constrains visual odometry. On the other hand, the entire line segment detected by the visual subsystem overcomes the limitation of the LiDAR subsystem, which can only perform the local calculation for geometric features. It adjusts the direction of linear feature points and filters out outliers, leading to a higher accurate odometry system. Finally, we employ a module to detect the subsystem's operation, providing the LiDAR subsystem's output as a complementary trajectory to our system while visual subsystem tracking fails. The evaluation results on the public dataset M2DGR, gathered from ground robots across various indoor and outdoor scenarios, show that our system achieves more accurate and robust pose estimation compared to current state-of-the-art multi-modal methods.
\end{abstract}
