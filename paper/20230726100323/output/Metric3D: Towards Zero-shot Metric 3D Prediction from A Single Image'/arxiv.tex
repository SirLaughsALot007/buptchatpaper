\documentclass[10pt,twocolumn,letterpaper]{article}

\usepackage{./ICCV2023/iccv}
\usepackage{times}
\usepackage{epsfig}
\usepackage{graphicx}
\usepackage{amsmath}
\usepackage{amssymb}

\usepackage{graphicx}
\usepackage{amsmath}
\usepackage{amssymb}
\usepackage{booktabs}

\usepackage{threeparttable}
\usepackage[vlined, ruled, linesnumbered]{algorithm2e}
\usepackage{multirow}
\usepackage{booktabs}
\usepackage{bbm}
\usepackage{pifont}
\usepackage{enumitem}

\usepackage{eucal,nicefrac}
\usepackage{subcaption}
\usepackage{gensymb}




\usepackage[margin=4pt,font=footnotesize,labelfont=bf,labelsep=endash,tableposition=top]{caption}


\usepackage[pagebackref=true,breaklinks=true,letterpaper=true,colorlinks,bookmarks=false]{hyperref}



\usepackage[capitalize]{cleveref}
\crefname{section}{Sec.}{Secs.}
\Crefname{section}{Section}{Sections}
\Crefname{table}{Table}{Tables}
\crefname{table}{Table}{Tables}

\iccvfinalcopy %

\def\iccvPaperID{5522} %
\def\httilde{\mbox{\tt\raisebox{-.5ex}{\symbol{126}}}}

\ificcvfinal\pagestyle{empty}\fi

\begin{document}

\title{Metric3D: Towards Zero-shot Metric 3D Prediction from A Single Image}


\def\SP{~~}

\author{
Wei Yin$^{1 \ast}$,
\SP
Chi Zhang$^2$\thanks{Equal contributions.}, 
\SP 
Hao Chen$^3$\thanks{Corresponding author.},
\SP
Zhipeng Cai$^4$,
\SP 
Gang Yu$^2$,
\SP 
Kaixuan Wang$^1$, \\
\SP 
Xiaozhi Chen$^1$,
\SP 
Chunhua Shen$^3$
\\[0.1325cm]
$ ^1$ DJI Technology
\SP ~~~
$ ^2 $ Tencent
\SP ~~~
$ ^3$ Zhejiang University
\SP ~~~
$ ^4$ Intel Labs
\\
{e-mail: $\tt\small ^1 \{yvan.yin, halfbullet.wang, xiaozhi.chen\}@dji.com;$ }\\
$ \tt\small^2\{johnczhang, skicyyu\}@tencent.com;$ \\
$ \tt\small ^3 haochen.cad@zju.edu.cn, chunhua@me.com;$ 
$\tt\small ^4 zhipeng.cai@intel.com$ 
}




\makeatletter
\let\@oldmaketitle\@maketitle%
\renewcommand{\@maketitle}{\@oldmaketitle%
 \centering
    \includegraphics[width=0.95\textwidth]{./files/front_img.pdf}
     \captionof{figure}{\textbf{
     Illustration 
     and applications of our metric 3D reconstruction
     method}.
     Top (metrology):  we use two phones (iPhone 12 and an Android phone) to capture the scene and measure the size of tables. With the photos' metadata, we perform 3D metric reconstruction and then measure tables' sizes (marked in red), which are very close to the ground truth (marked in blue). In contrast, the recent method LeReS~\cite{leres} performs much worse and is unable to predict metric 3D by design.
     Bottom  (dense SLAM mapping): existing SOTA mono-SLAM methods usually face scale drift problems (see the red arrows) in large-scale scenes and are unable to achieve the metric scale, while, naively inputting our metric depth, Droid-SLAM~\cite{teed2021droid} can recover much more accurate trajectory and perform the \textit{metric} dense mapping (see the red measurements). 
     Note that all testing data are unseen to our model.
}
    \label{Fig: first page fig.}
    \bigskip}                   %
\makeatother

\maketitle


\def\PWN{{\rm PWN}}
\def\VNL{{\rm VNL}}
\def\RPNL{{\rm RPNL}}


\begin{abstract}

Reconstructing accurate 3D scenes from images is a long-standing vision task. Due to the ill-posedness of the single-image reconstruction problem, most well-established methods are built upon multi-view geometry. State-of-the-art (SOTA) monocular metric depth estimation methods can only handle a single camera model and are unable to perform mixed-data training due to the metric ambiguity. Meanwhile, SOTA monocular methods trained on large mixed datasets achieve zero-shot generalization by learning affine-invariant depths, which cannot recover real-world metrics. In this work, we show that the key to a zero-shot single-view metric depth model lies in the combination of large-scale data training and resolving the metric ambiguity from various camera models. We propose a canonical camera space transformation module, which explicitly addresses the ambiguity problems and can be effortlessly plugged into existing monocular models. Equipped with our module, monocular models can be stably trained over $8$ million of images with thousands of camera models, resulting in zero-shot generalization to in-the-wild images with unseen camera settings. 

\textbf{ Experiments demonstrate SOTA performance of our method on $7$ zero-shot benchmarks.
Notably, our method won the championship in the }\href{https://jspenmar.github.io/MDEC/}{2nd Monocular Depth Estimation Challenge}.
Our method enables the accurate recovery of metric 3D structures on randomly collected internet images, paving the way for plausible single-image metrology. The potential benefits extend to downstream tasks, which can be significantly improved by simply plugging in our model. 
For example, 
our model relieves the scale drift issues of monocular-SLAM (Fig.~\ref{Fig: first page fig.}), leading to high-quality metric scale dense mapping.  The code is available at \url{https://github.com/YvanYin/Metric3D}.
\end{abstract}

\section{Introduction}
\begin{figure}
  \vskip-1ex
  \centering
  \includegraphics[width=0.5\textwidth]{images/robot2.pdf}
  \vskip-2ex
  \caption{Our method projects the geometric features (extracted from the sensor camera and LiDAR data) onto the fusion frame (the spherical coordinate system). Through a series of optimizations, a more accurate trajectory and map are obtained with the reconstructed line.} 
  \label{fig:overview_intro}
  \vskip -3ex
\end{figure}
%
In the field of mobile robots and autonomous navigation agents, there has been a notable increase of interest among different studies \cite{pandey2017mobile, zhu2021deep, patle2019review}. 
They are employed for a variety of purposes, including independent living for the elderly~\cite{cortes2007assistive, sasaki2009human}, guiding and assisting shoppers in large retail spaces~\cite{orciuoli2015agent, qu2021outline}, and facilitating delivery services in outdoor industrial or commercial areas~\cite{abbenseth2017cloud, lopez2017predictive}. When a robot enters a new scene or encounters an environment with updated details, it carefully plans actions to explore unknown or uncertain areas. The goal is to gather valuable information about the new terrain and improve the accuracy of its reconstructed map. Thereby, the robot utilizes the measurements provided by its sensors to reconstruct the map and perform localization through SLAM. 

Tracking failure in challenging environments has hindered the widespread adoption of mobile robots. While visual SLAM \cite{forster2014svo, mur2015orb, mur2017orb} offers superior motion estimation, it tends to fail in extreme lighting conditions and outdoor environments. Similarly, LiDAR SLAM~\cite{deschaud2018imls, shan2018lego, zhang2014loam} is highly reliable for tracking motion in large-scale scenes but faces challenges in degraded cases. Traditional SLAM methods rely heavily on a single source, which may be prone to failure in adverse scenarios or due to equipment issues. In recent years, significant approaches~\cite{chou2021efficient, shan2021lvi, zhu2021camvox} have been developed to integrate measurements from multiple sensors to overcome the limitations of mono-sensor algorithms. However, some SLAM systems~\cite{chou2021efficient} combine camera, IMU, and LiDAR to recalibrate the extrinsic parameters which can be quite computationally expensive. Therefore, our research focuses on the development of a camera-LiDAR framework. The primary goal of this framework is to minimize sensor costs for mobile agents while conserving computational resources. Moreover, it aims to provide highly accurate mapping and localization in a variety of scenarios.

%
In practical situations, devices often oscillate when in motion. This leads to less accurate extrinsic calibration between the camera and the LiDAR. The fusion performance~\cite{graeter2018limo}, which relies solely on the association of feature points, is highly susceptible to the misalignment error caused by this oscillation. Consequently, inaccurate depth estimates are obtained, compromising the overall accuracy of the system.
Other SLAM methods~\cite{fang2020visual, lee2021plf, pumarola2017pl} that depend on geometric features prove that lines are more reliable and stable. Therefore, our work is exclusively concentrated on geometric-level fusion to combine the monocular camera and LiDAR sensors.

In this paper, we propose a multi-modal SLAM system that tightly integrates parallel monocular visual SLAM and LiDAR SLAM subsystems. This design ensures that if one subsystem fails, the other subsystem continues to operate, providing a more robust system for robot navigation. Moreover, we utilize more stable linear and planar features to minimize the impact of inaccurate extrinsic calibration. In Figure~\ref{fig:overview_intro}, each subsystem utilizes perceptual data from the mobile robot's sensors (a camera and a LiDAR sensor) to extract geometric features. We align the features from both subsystems in terms of temporality, spatiality, and dimensionality through a unified reference fusion frame. After fusion, the reconstructed features contribute as new landmarks to the initial pose estimation in the visual subsystem. They are utilized as additional optimization parameters in the back end, where the endpoint and direction are used to constrain the visual odometry. In the LiDAR subsystem, the direction of the geometric features is adjusted by the detected line from the visual system, reducing the probability of outliers during registration.

We evaluate our approach on the M2DGR dataset \cite{yin2021m2dgr}, which consists of various indoor and outdoor environments commonly encountered in mobile robot applications. By comparing with our visual baselines Structure PLP-SLAM~\cite{shu2022structure}, LiDAR baseline MULLS~\cite{pan2021mulls}, and multi-modal algorithms LVI-SAM~\cite{shan2021lvi} and ORB-SLAM3~\cite{campos2021orb}, our SLAM method achieves superior accuracy and robustness in various challenging environments. It is particularly suitable for low-cost mobile robots in various applications.

To summarize, we present the following contributions:

\begin{compactitem}
\item We build a fusion framework -- a spherical coordinate system -- to maintain spatial and temporal consistency. This framework integrates the geometric features from the visual subsystem and the LiDAR subsystem.

\item In the LiDAR subsystem, we optimize the linear directions to improve registration efficiency and accuracy, while increasing the probability of optimized points in its local map.

\item We reconstruct more lines for pose estimation in the visual subsystem. In the back end of the visual subsystem, we propose a new optimization term, \textit{i.e.}, line direction term, and more fusion lines as optimization parameters to constrain the trajectory.

\end{compactitem}

\section{Related Work}
\noindent\textbf{3D reconstruction from a single image.} 
Reconstructing various objects from a single image has been well studied~\cite{barron2014shape, wang2018pixel2mesh, wu2018learning}. They can produce high-quality 3D models of cars, planes, tables, and human body~\cite{saito2019pifu, saito2020pifuhd}. The main challenge is how to best recover objects' details, how to represent them with limited memory, and how to generalize to more diverse objects. However, all these methods rely on learning priors specific to a certain object class or instance, typically from 3D supervision, and can therefore not work for full scene reconstruction. Apart from these reconstructing objects works, several works focus on scene reconstruction~\cite{Xu_2023_ICCV} from a single image. Saxena~\etal~\cite{saxena2008make3d} construct the scene based on the assumption that the whole scene can be segmented into several small planes. With planes' orientation and location, the 3D structure can be represented. Recently, LeReS~\cite{leres} propose to use a strong monocular depth estimation model to do scene reconstruction. However, they can only recover the shape up to a scale. 
Zhang~\etal~\cite{Zhang_2023_ICCV} recently  propose a zero-shot geometry-preserving depth estimation model that is capable of making depth predictions up to an unknown scale, without requiring  scale-invariant depth annotations for training. 
In contrast to these works, our method can recover the metric 3D structure.



\noindent\textbf{Supervised monocular depth estimation.}
After several benchmarks~\cite{silberman2012indoor, Geiger2013IJRR} are established, neural network based
methods~\cite{yuan2022new, Yin2019enforcing, bhat2021adabins} have dominated since then. Several approaches regress the continuous depth from the aggregation of information in an image~\cite{eigen2014depth}. As depth distribution corresponding to different RGBs can vary to a large extent, some methods~\cite{Yin2019enforcing, 
bhat2021adabins}
discretize the depth and formulate this problem to a classification~\cite{yin2021virtual},  
which often achieves better performance. %
The generalization issue of deep models for 3D metric recovery   
is related to two problems. 
The first one is to generalize to diverse scenes, while 
the other 
one is how to predict accurate metric information under various camera settings. The first problem has been well addressed by recent methods. Some works~\cite{
xian2020structure, xian2018monocular, yin2021virtual} propose to construct a large-scale relative depth dataset, such as DIW~\cite{chen2016single} and OASIS~\cite{chen2020oasis}, and then they target learning the relative relations. However, the relative depth loses geometric structure information.
To improve the recovered geometry quality, 
learning affine-invariant depth methods, such as MiDaS~\cite{Ranftl2020}, LeReS~\cite{leres}, and HDN~\cite{ 
zhang2022hierarchical} are proposed. 
By mixing large-scale data, 
state-of-the-art performance and the generalization over scenes are improved continuously. 
Note that by design, these methods are unable to recover 
the metric information.
How to achieve both strong generalization and accurate metric information over diverse scenes is the key problem that
we attempt to tackle. 

\noindent\textbf{Large-scale data training.}
Recently, various natural language problems and computer vision problems~\cite{
yin2022devil, radford2021learning, lambert2020mseg} have achieved impressive progress with large-scale data training. CLIP~\cite{radford2021learning} is a promising classification model, which is trained on billions of paired image and language descriptions data. It achieved state-of-the-art performance over several classification benchmarks by zero-shot testing. 
For depth prediction, large-scale data training has been widely applied. Ranft~\etal~\cite{Ranftl2020} mixed over 2 million data in training, LeReS~\cite{yin2022towards} collected over $300$ thousands data, Eftekhar~\etal~\cite{eftekhar2021omnidata} also merged millions of data to build a strong depth prediction model. %







\section{Method}












\begin{figure*}[]
\centering
\includegraphics[width=0.99\textwidth]{./files/pipeline3}
\vspace{-1 em}
\caption{\textbf{Pipeline.} 
Given an input image $I$, we first transform it to the canonical space using CSTM. The transformed image $I_c$ is fed into a standard depth estimation model to produce the predicted metric depth $D_c$ in the canonical space. During training, $D_c$ is supervised by a GT depth $D^*_c$ which is also transformed into the canonical space. In inference, after producing the metric depth $D_c$ in the canonical space, we perform a de-canonical transformation to convert it back to the space of the original input $I$. The canonical space transformation and de-canonical transformation are executed using camera intrinsics.}
\label{fig: pipeline}
\vspace{-1em}
\end{figure*}

\noindent\textbf{Preliminaries.}
We consider the pin-hole camera model with intrinsic parameters are:
[[$\nicefrac{\hat{f}}{\delta}, 0, u_{0}$], [$0,  \nicefrac{\hat{f}}{\delta},  v_{0}$], [$0, 0,  1$]],
where $\hat{f}$ is the 
focal length (in micrometers), $\delta$ is the pixel size
(in micrometers), and
$(u_{0}, v_{0})$ is the principle center. $f = \nicefrac{\hat{f}}{\delta}$ is the pixel-represented focal length used in vision algorithms.








\subsection{Metric Ambiguity Analysis}\label{sec:ambiguity}

Fig.~\ref{fig: inspiration} presents an example of photos taken by different cameras and at 
different distances. Only from the image's appearance, 
one may think 
the last two photos are taken at %
a 
similar location by the same camera. %
In fact, due to different focal lengths, 
these %
are captured at different locations.
Thus, 
camera %
intrinsic parameters are 
critically 
important for the metric estimation from a single image, as otherwise, the problem is \textit{ill posed}. 
To avoid such metric ambiguity, recent %
methods, such as MiDaS~\cite{Ranftl2020} and LeReS~\cite{leres}, decouple the metric from the supervision and %
compromise learning the affine-invariant depth.



\begin{figure}[!bt]
\centering
\includegraphics[width=0.47\textwidth]{./files/inspiration}
\vspace{-0.5 em}
\caption{\textbf{Photos of a chair captured at different distances with different cameras}. The first two photos are captured at the same distance but with different cameras, 
while the last one is taken at a closer distance with the same camera as the first one.}
\label{fig: inspiration}
\vspace{-1.5em}
\end{figure}

Fig.~\ref{fig: pinhole camera} (A) shows a simple pin-hole perspective projection. Object $A$ locating at $d_{a}$ is projected to $A'$. 
Based on the principle of similarity, we %
have the equation:
\begin{equation}
\vspace{-1em}
    d_{a} = \hat{S} \Bigl[\frac{\hat{f}}{\hat{S'}}\Bigr]= \hat{S}\cdot \alpha
\label{eq: similarity}
\end{equation}
where $\hat{S}$ and $\hat{S'}$ are the real and \textit{imaging} size respectively. $\hat{\cdot}$ denotes variables are in the physical metric (\textit{e.g.}, millimeter). To %
recover 
$d_{a}$ from a single image, focal length, imaging size of the object, and real-world object size %
must be available. 
Estimating the focal length 
from a single image is a %
challenging 
and ill-posed problem. Although several methods~\cite{leres, hold2018perceptual} have explored,
the accuracy 
is still far from being satisfactory. 
Here, we simplify the problem by assuming 
the focal length of a training/test image is available. 
In contrast, understanding the imaging size is much easier for a neural network. To obtain the real-world object size, a neural network %
needs to 
understand the semantic scene layout and the object, at which a neural network excels. 
We %
define 
$\alpha = \nicefrac{\hat{f}}{\hat{S'}} $, so $d_{a}$ is proportional to $\alpha$. 
\begin{figure}[!b]
\vspace{-2em}
\centering
\includegraphics[width=0.5\textwidth]{./files/pinhole_difference_distances}
\caption{\textbf{Pinhole camera model}. (A) Object $A$ at the distance $d_{a}$ is projected to the image plane. (B) Using two cameras to capture the car. The left one has a larger pixel size. Although the projected imaging sizes are the same, the pixel-represented images (resolution) are different.}
\label{fig: pinhole camera}
\vspace{-0.5em}
\end{figure}

We make the following observations regarding sensor size, pixel size, and focal length.


\noindent\textbf{O1: Sensor size and pixel size do not affect the metric depth estimation.} 
Based on the perspective projection (Fig.~\ref{fig: pinhole camera} (A)), the sensor size only affects the field of view (FOV) and is irrelevant to $\alpha$, thus does not affect the metric depth estimation. 
For the pixel size, %
we assume two cameras with different pixel sizes ($\delta_{1} = 2\delta_{2}$) but the same focal length $\hat{f}$ to capture the same object locating at $d_{a}$. %
Fig.~\ref{fig: pinhole camera} (B) shows their captured photos.
According to the preliminaries, %
the pixel-represented focal length $f_{1} = \frac{1}{2} f_{2}$. 
As the second camera has a smaller pixel size, although in the same projected imaging size $\hat{S'}$, the pixel-represented image resolution is $S'_{1} = \frac{1}{2} S'_{2}$. According to Eq.~\eqref{eq: similarity}, $\frac{\hat{f}}{\delta_{1}\cdot S'_{1}} = \frac{\hat{f}}{\delta_{2}\cdot S'_{2} }$, i.e. $\alpha_1 = \alpha_2$, so $d_{1} = d_{2}$. Therefore, different camera sensors %
would 
not affect the metric depth estimation.


\noindent\textbf{O2: The focal length is vital for %
metric depth estimation}. Fig.~\ref{fig: inspiration} shows the metric ambiguity issue caused by the unknown focal length. 
Fig.~\ref{fig: confusion} %
illustrates 
this. If two cameras ($\hat{f}_{1} = 2\hat{f}_{2}$) are at distances $d_{1} = 2d_{2}$, the imaging sizes on cameras are the same. Thus, only from the appearance, %
the network will be confused when supervised with different labels.
Based on this observation, we propose a canonical camera transformation method to solve the supervision and image appearance conflicts.

\begin{figure}[!bt]
\centering
\includegraphics[width=0.48\textwidth]{./files/confusion}
\caption{\textbf{Illustration of two cameras with different focal length} at different distance. As $f_1=2f_2$ and $d_1=2d_2$, 
$A$ is projected 
to two image planes with the same imaging size (i.e. $A^{'}_1 = A^{'}_2$).
}
\label{fig: confusion}
\vspace{-2em}
\end{figure}




\subsection{Canonical Camera Transformation}
The core idea is to set up a canonical camera space ($(f_{x}^{c}, f_{y}^{c})$, $f_{x}^{c}=f_{y}^{c}=f^{c}$ in experiments) and transform all training data to this space. 
Consequently, all data can roughly be regarded as captured by the canonical camera.
We propose two transformation methods, i.e. either transforming the input image ($\mathbf{I}\in\mathbb{R}^{H \times W \times 3}$) or the ground-truth (GT) label ($\mathbf{D}\in\mathbb{R}^{H \times W}$). The original intrinsics are $\{f, u_{0}, v_{0}\}$.

\noindent\textbf{Method1: transforming depth labels (CSTM\_label).}
Fig.~\ref{fig: inspiration}'s ambiguity is for depths. 
Thus our first method directly transforms the ground-truth depth labels to solve this problem.  Specifically, we scale the ground-truth depth ($\mathbf{D}^{*}$) with the ratio $\omega_d = \frac{f^{c}}{f}$ in training, \textit{i.e.},  $\mathbf{D}^{*}_{c} = \omega_d \mathbf{D}^{*}$. The original camera model is transformed to $\{f^{c}, u_{0}, v_{0}\}$. In inference, the predicted depth ($\mathbf{D}_{c}$) is in the canonical space and needs to perform a de-canonical transformation to recover the metric information, \textit{i.e.}, $\mathbf{D} = \frac{1}{\omega_d}\mathbf{D}_{c}$. Note the input $\mathbf{I}$ does not perform any transformation, \textit{i.e.},  $\mathbf{I}_c = \mathbf{I}$.

\noindent\textbf{Method2: transforming input images (CSTM\_image).}
From another view, the ambiguity is caused by the similar image appearance. Thus this method is to transform the input image to simulate the canonical camera imaging effect.
Specifically, the image $\mathbf{I}$ is resized with the ratio $\omega_r=\frac{f^{c}}{f}$, \textit{i.e.},
$\mathbf{I}_{c} = \mathcal{T}(\mathbf{I}, \omega_r)$, where $\mathcal{T}(\cdot)$ denotes image resize. The optical center is resized, thus the canonical camera model is $\{f^{c}, \omega_r u_{0}, \omega_r v_{0}\}$. The ground-truth labels are resized without any scaling, \textit{i.e.}, 
$\mathbf{D}^{*}_{c} = \mathcal{T}(\mathbf{D}^*, \omega_r)$. In inference, the de-canonical transformation is to resize the prediction to the original size without scaling, \textit{i.e.}, $\mathbf{D} = \mathcal{T}(\mathbf{D}_{c}, \frac{1}{\omega_r})$.

Fig.~\ref{fig: pipeline} shows the pipeline. After performing either transformation, we randomly crop a patch for training. %
The cropping only adjusts the FOV and the optical center, %
thus not causing any metric ambiguity issues. In the labels transformation method $\omega_r = 1$ and $\omega_d=\frac{f^c}{f}$, while $\omega_d = 1$ and $\omega_r=\frac{f^c}{f}$ in the images transformation method. The training objective is as follows:
\begin{equation}
    \min_{\theta}\left | \mathcal{N}_{d}(\mathbf{I}_{c}, \theta) - \mathbf{D}^{*}_{c} \right| 
\label{eq: robust metric depth}
\vspace{-0.5 em}
\end{equation}
where 
$\theta$ is the network's ($\mathcal{N}_{d}(\cdot)$) parameters, $\mathbf{D}^{*}_{c}$ and $\mathbf{I}_{c}$ are transformed ground-truth depth labels and images.




Mix-data training is an effective way to boost generalization. 
We collect $11$ datasets for training, see 
the supplementary materials for details. In the mixed data, over 10K different cameras are included. %
All collected training data have included paired camera intrinsic parameters, which are %
used in our canonical transformation module. 


\noindent\textbf{Supervision.} To further boost the performance, we propose a random proposal normalization loss (RPNL). The scale-shift invariant loss~\cite{Ranftl2020, leres} is widely applied for the affine-invariant depth estimation, which decouples the depth scale to emphasize the single image distribution. However, such normalization based on the whole image inevitably squeezes the fine-grained depth difference, particularly in close regions. Inspired by this, we propose to randomly crop several patches ($p_{i(i=0,...,M)} \in \mathbb{R}^{h_i \times w_i}$) from the ground truth $\mathbf{D}^{*}_c$ and the predicted depth $\mathbf{D}_c$. Then we employ the median absolute deviation normalization~\cite{singh2019investigating} for paired patches. By normalizing the local statistics, we can enhance local contrast. The loss function is as follows:
\begin{eqnarray}\nonumber
    L_{\RPNL} = \frac{1}{MN} \sum_{p_i}^{M}\sum_{j}^{N} \lvert \frac{d^{*}_{p_i, j} - \mu(d^{*}_{p_i, j})}{\frac{1}{N}\sum_{j}^{N} \left |d^{*}_{p_i, j} - \mu(d^{*}_{p_i, j}) \right |} - \\
    \frac{d_{p_i, j} - \mu(d_{p_i, j})}{\frac{1}{N}\sum_{j}^{N} \left | d_{p_i, j} - \mu(d_{p_i, j}) \right |} \rvert
\label{eq: RPNL}
\end{eqnarray}
where $d^*\in \mathbf{D}^*_c$ and $d \in \mathbf{D}_c$ are the ground truth and predicted depth respectively. $\mu(\cdot)$ and is the median of depth. $M$ is the number of proposal crops, which is set to 32. During training, proposals are randomly cropped from the image by 
$0.125$ to $0.5$ of the original size. Furthermore, several other losses are employed, including the scale-invariant logarithmic loss~\cite{eigen2014depth} $L_{silog}$, pair-wise normal regression loss~\cite{leres}$L_{\PWN}$, virtual normal loss~\cite{yin2021virtual} $L_{\VNL}$. Note $L_{silog}$ is a variant of L1 loss.  The overall losses are as follows.
\begin{eqnarray}\nonumber
    L = L_{\PWN} + L_{\VNL} + L_{silog} + L_{\RPNL}.
\label{eq: losses}
\end{eqnarray}
\vspace{-2 em}



\section{Experiments}

\noindent\textbf{Dataset details.}
\label{sec:data}
We collect $11$ public RGB-D datasets, and over $8$ million data for training. It spreads over diverse indoor and outdoor scenes. Note that all datasets have provided %
camera intrinsic parameters. Apart from the test split of training datasets, we collect $7$ unseen datasets for robustness and generalization evaluation. Details of employed data are reported in the supplementary materials.  



\noindent\textbf{Implementation details.}
We employ an UNet architecture with the ConvNext-large~\cite{liu2022convnet} backbone. ImageNet-22K pre-trained weights are used for initialization. We use AdamW with a batch size of $192$, an initial learning rate $0.0001$ for all layers, and the polynomial decaying method with the power of $0.9$. We train our final model on $48$ A100 GPUs for $500$K iterations. Following the DiverseDepth~\cite{yin2021virtual}, we balance all datasets in a mini-batch to ensure each dataset accounts for an almost equal ratio. During training, images are processed by the canonical camera transformation module, flipped horizontally with a $50\%$ chance, and then randomly cropped into 
$512 \times 960$ pixels. For the ablation experiments, training settings are different as we sample $5000$ images from each dataset for training. We trained on $8$ GPUs for $150$K iterations.










\noindent\textbf{Evaluation details.}
a) To show the robustness of our metric depth estimation 
method, we test on 8 zero-shot benchmarks, including NYUv2~\cite{silberman2012indoor}, KITTI~\cite{Geiger2013IJRR}, NuScenes~\cite{caesar2020nuscenes}, 7-scenes~\cite{shotton2013scene}, iBIMS-1~\cite{koch2018evaluation}, DIODE~\cite{vasiljevic2019diode}, ETH3D~\cite{schops2017multi}. Following previous works~\cite{yuan2022new}, absolute relative error (AbsRel),  the accuracy under threshold ($\delta_{i} < 1.25^{i}, i=1, 2, 3$), root mean squared error (RMS), root mean squared error in log space (RMS\_{log}), and log10 error (log10) metrics are employed. 
b) Furthermore, %
we also follow current affine-invariant depth benchmarks~\cite{leres, zhang2022hierarchical} (Tab. \ref{Table: generalization evaluation.}) to evaluate the generalization ability on $5$ zero-shot datasets, \textit{i.e.},  NYUv2, DIODE, ETH3D, ScanNet~\cite{dai2017scannet}, and KITTI. We mainly compare with large-scale data trained models. Note that in this benchmark we follow existing methods to apply the scale shift alignment before evaluation. 
c) To evaluate our metric 3D reconstruction quality, we randomly sample 9 unseen scenes from NYUv2 and use colmap~\cite{schoenberger2016mvs} to obtain the camera poses for multi-frame reconstruction. Chamfer $l_1$ distance and the F-score~\cite{knapitsch2017tanks} are used to evaluate the reconstruction accuracy. 
d) In dense-SLAM experiments, following Li~\etal~\cite{li2021generalizing}, we test on the KITTI odometry benchmark~\cite{Geiger2013IJRR} and evaluate the average translational RMS drift ($\%, t_{rel}$) and rotational RMS drift ($\degree/100m, r_{rel}$) errors~\cite{Geiger2013IJRR}.
Note that all these depth and reconstruction evaluations use the same trained model. 

\subsection{Zero-shot Generalization %
Test
}

\begin{table}[!t]
\caption{Quantitative comparison on NYUv2 and KITTI benchmarks. Both datasets are unseen to our model, but we can achieve comparable performance with state-of-the-art methods.}
\vspace{-1 em}
\scalebox{0.67}{
\begin{tabular}{r |cccccc}
\toprule[1pt]
\multicolumn{7}{c}{\textbf{NYUv2 Benchmark}} \\ \hline
\multirow{1}{*}{Method} & $\boldsymbol{\delta_{1}}$$\uparrow$ & $\boldsymbol{\delta_{2}}$$\uparrow$ & $\boldsymbol{\delta_{3}}$$\uparrow$ & \textbf{AbsRel}$\downarrow$ & \textbf{log10}$\downarrow$ & \textbf{RMS}$\downarrow$  \\ \hline
Li \etal.~\cite{li2017two}               & $0.788$    & $0.958$    & $0.991$  & $0.143$   & $0.063$    & $0.635$     \\
Laina \etal.~\cite{laina2016deeper}      & $0.811$    & $0.953$    & $0.988$  & $0.127$   & $0.055$    & $0.573$       \\
VNL ~\cite{Yin2019enforcing}            & $0.875$   & $0.976$    & $0.994$  & $0.108$   & $0.048$    & $0.416$    \\ 
TrDepth~\cite{yang2021transformers}     & $0.900$   & $0.983$    & $0.996$  & $0.106$  & $0.045$     & $0.365$   \\
Adabins~\cite{bhat2021adabins}         & $0.903$    & ${0.984}$  & $\underline{0.997}$  & $0.103$  & $0.044$     & $0.364$    \\
NeWCRFs~\cite{yuan2022new}              
& ${0.922}$  & $\boldsymbol{0.992}$  & $\boldsymbol{0.998}$ 
& $0.095$  & $0.041$     & $\underline{0.334}$   \\ \hline
Ours CSTM\_image    
& $\boldsymbol{0.925}$  & $0.983$  & $0.994$ 
& $\underline{0.092}$   & $\underline{0.040}$   & ${0.341}$  \\
Ours CSTM\_label    
& $\underline{0.944}$  & $\underline{0.986}$  & $0.995$ 
& $\boldsymbol{0.083}$   & $\boldsymbol{0.035}$   & $\boldsymbol{0.310}$  \\\hline
\hline
\multicolumn{7}{c}{\textbf{KITTI Benchmark}} \\ \hline \hline
\multirow{1}{*}{Method} & $\boldsymbol{\delta_{1}}$$\uparrow$ & $\boldsymbol{\delta_{2}}$$\uparrow$ & $\boldsymbol{\delta_{3}}$$\uparrow$ & \textbf{AbsRel} $\downarrow$ & \textbf{RMS} $\downarrow$ & \textbf{RMS\_log} $\downarrow$ \\ \hline
Guo \etal \cite{guo2018learning}  & $0.902$  & $0.969$  & $0.986$ & $0.090$ & $3.258$ & $0.168$    \\
VNL~\cite{Yin2019enforcing} & ${0.938}$   & ${0.990}$   & ${0.998}$  & ${0.072}$  & $3.258$      & ${0.117}$    \\ 
TrDepth~\cite{yang2021transformers}   & $0.956$  & $0.994$  & $0.999$   & $0.064$  & $2.755$  & $0.098$  \\
Adabins~\cite{bhat2021adabins} & $0.964$  & $0.995$  & $0.999$   & $\underline{0.058}$  & $2.360$  & $0.088$  \\
NeWCRFs~\cite{yuan2022new} 
& $\boldsymbol{0.974}$  & $\boldsymbol{0.997}$  & $\underline{0.999}$   & $\boldsymbol{0.052}$  & $\boldsymbol{2.129}$  & $\boldsymbol{0.079}$  \\ \hline
Ours CSTM\_image & 
$\underline{0.967}$   & $\underline{0.995}$   & $\boldsymbol{0.999}$  
& $0.060$  & ${2.843}$     & $\underline{0.087}$    \\ 
Ours CSTM\_label 
& ${0.964}$   & ${0.993}$   & ${0.998}$  
& $0.058$  & ${2.770}$     & ${0.092}$    \\ \hline
\toprule[1pt]
\end{tabular}\newline}
\label{table:errors cmp on NYUD-V2}
\vspace{-2 em}
\end{table}






\begin{table*}[]
\renewcommand\arraystretch{1.1}
\caption{Quantitative comparison of 3D scene reconstruction with LeReS~\cite{leres}, DPT~\cite{ranftl2021vision}, %
RCVD~\cite{kopf2021rcvd}, 
SC-DepthV2~\cite{bian2021tpami}, and a learning-based MVS method (DPSNet~\cite{im2019dpsnet}) on 9 unseen NYUv2 scenes. Apart from DPSNet and ours, other methods have to align the scale with ground truth depth for each frame. As a result, our reconstructed 3D scenes achieve the best performance.}
\vspace{-1 em}
\centering
\resizebox{.98\linewidth}{!}{%
  \centering
  \small 
  \setlength{\tabcolsep}{0.5mm}{\begin{tabular}{@{} r |rc|rc|rc|rc|rc|rc|rc|rc|rc@{}}
    \toprule
    \multirow{2}{*}{Method} & \multicolumn{2}{c|}{Basement\_0001a} & \multicolumn{2}{c|}{Bedroom\_0015} & \multicolumn{2}{c|}{Dining\_room\_0004} & \multicolumn{2}{c|}{Kitchen\_0008} & \multicolumn{2}{c|}{Classroom\_0004} & \multicolumn{2}{c|}{Playroom\_0002}  & \multicolumn{2}{c|}{Office\_0024} & \multicolumn{2}{c|}{Office\_0004} & \multicolumn{2}{c}{Dining\_room\_0033}\\
      & C-$l_1$$\downarrow$ & F-score $\uparrow$ & C-$l_1$$\downarrow$ & F-score $\uparrow$ & C-$l_1$$\downarrow$ & F-score $\uparrow$ & C-$l_1$$\downarrow$ & F-score $\uparrow$ & C-$l_1$$\downarrow$ & F-score $\uparrow$ & C-$l_1$$\downarrow$ & F-score $\uparrow$ &
      C-$l_1$$\downarrow$ & F-score $\uparrow$ & C-$l_1$$\downarrow$ & F-score $\uparrow$ & C-$l_1$$\downarrow$ & F-score $\uparrow$\\ \hline
    RCVD~\cite{kopf2021rcvd} & 0.364 & 0.276 
            & 0.074 & 0.582 &
             0.462 & 0.251 &
             0.053 & 0.620 &
              0.187 & 0.327 &
             0.791 & 0.187 &
             0.324 & 0.241  &
             0.646 & 0.217 &
             0.445 & 0.253 \\

    SC-DepthV2~\cite{bian2021tpami}  & 0.254 & 0.275 &
             0.064 & 0.547 &
             0.749 & 0.229 &
             0.049 & 0.624 &
              0.167 & 0.267 &
             0.426 & 0.263 &
             0.482 & 0.138  &
             0.516 & 0.244 &
             0.356 &0.247 \\

    DPSNet~\cite{im2019dpsnet} & 0.243 & 0.299 &
             0.195 & 0.276 &
             0.995 & 0.186 &
             0.269 & 0.203 &
             0.296 & 0.195 &
             0.141 & 0.485 &
             0.199 & 0.362  &
             0.210 & 0.462 &
             0.222 & 0.493 \\
              
    DPT~\cite{leres} & 0.698 & 0.251 &
             0.289 & 0.226 &
             0.396 & 0.364 &
             0.126 & 0.388 &
             0.780 & 0.193 & 
             0.605 & 0.269 &
             0.454 & 0.245  &
             0.364 & 0.279 &
             0.751 & 0.185 \\  
    LeReS~\cite{leres} & 0.081 & 0.555 &
             0.064 & 0.616 &
             0.278 & 0.427 &
             0.147 & 0.289 &
             \textbf{0.143} & \textbf{0.480} &
             0.145 & 0.503 &
             0.408 & 0.176  &
             0.096 & 0.497 &
             0.241 & 0.325 \\  \hline
    Ours & \textbf{0.042} & \textbf{0.736} &
             \textbf{0.059} & \textbf{0.610} &
             \textbf{0.159} & \textbf{0.485} &
             \textbf{0.050} & \textbf{0.645} &
             0.145 & 0.445 &
             \textbf{0.036} & \textbf{0.814} &
             \textbf{0.069} & \textbf{0.638}  &
             \textbf{0.045} & \textbf{0.700} &
             \textbf{0.060} & \textbf{0.663} \\

    \bottomrule
  \end{tabular}}}
  \label{tab: NYUD reconstruction cmp.}
\end{table*}

\begin{figure*}[]
\centering
\includegraphics[width=0.95\textwidth]{./files/3dreconstruction.pdf}
\vspace{-1 em}
\caption{\textbf{Reconstruction of zero-shot scenes with multiple views.} We sample several NYUv2 scenes for 3D reconstruction comparison. As our method can predict accurate metric depth, thus all frame's predictions are  fused together for scene reconstruction. By contrast, LeReS~\cite{leres}'s depth is up to an unknown scale and shift, which causes noticeable distortions. DPSNet~\cite{im2019dpsnet} is a multi-view stereo method, which cannot work well on low-texture regions. }
\label{fig: visual nyud reconstruction cmp.}
\vspace{-1em}
\end{figure*}




\noindent\textbf{Evaluation on metric depth benchmarks.} To evaluate the accuracy of predicted metric depth, firstly,  we compare with state-of-the-art (SOTA) metric depth prediction methods on NYUv2~\cite{silberman2012indoor}, KITTI~\cite{geiger2012we}.
We use the same model to do %
all evaluations. %
Results are reported in Tab.~\ref{table:errors cmp on NYUD-V2}. Without any fine-tuning or metric adjustment,  we can achieve comparable performance with SOTA methods, which are trained on benchmarks for hundreds of epochs. %







Furthermore, We collect $6$ unseen datasets to do more metric accuracy evaluation. These datasets contain a wide range of indoor and outdoor scenes, including rooms, buildings, and driving scenes. The camera models are also various, e.g. 7scenes has a short focal length (around 500), while ETH3D is 2000. We mainly compare with the SOTA metric depth estimation methods and take their NYUv2 and KITTI models for indoor and outdoor scenes evaluation respectively. From Tab. \ref{table: metric eval on more datasets.}, we observe that although 7Scenes is similar to NYUv2 and NuScenes is similar to KITTI, existing methods face a noticeable performance decrease. In contrast, our model is more robust. %

\begin{table*}[]
\centering
 \caption{Quantitative comparison with SOTA metric depth methods on $6$ unseen benchmarks. For SOTA methods, we use their NYUv2 and KITTI models for indoor and outdoor scenes evaluation respectively, while we use the same model for all zero-shot testing. }
 \vspace{-1 em}
 \resizebox{0.9\linewidth}{!}{%
\begin{tabular}{l|lll|lll}
\toprule[1pt]
\multirow{2}{*}{Method}        & DIODE(Indoor) & iBIMS-1 & 7Scenes      & DIODE(Outdoor)      & ETH3D      & NuScenes     \\
        & \multicolumn{3}{c|}{Indoor scenes (AbsRel$\downarrow$/RMS$\downarrow$)}  & \multicolumn{3}{c}{Outdoor scenes (AbsRel$\downarrow$/RMS$\downarrow$)} \\ \hline
Adabins~\cite{bhat2021adabins}  
         &  0.443 / 1.963       
         &0.212 / 0.901         
         & 0.218 / 0.428 
         &0.865 / 10.35                     
         &1.271 / 6.178            
         &0.445 / 10.658              \\
NewCRFs~\cite{yuan2022new}  
         &0.404 / 1.867               
         &0.206 / 0.861         
         &0.240 / 0.451 
         &0.854 / 9.228                     
         &0.890 / 5.011            
         &0.400 / 12.139              \\ \hline
Ours\_CSTM\_label    
        &\textbf{0.252} / \underline{1.440}               
        & \underline{0.160} / \textbf{0.521}         
        &  \textbf{0.183} / \textbf{0.363}   
        &\textbf{0.414} / \underline{6.934}                     
        & \underline{0.416} / \underline{3.017}          
        & \underline{0.154} / \underline{7.097}             \\
Ours\_CSTM\_image    
        & \underline{0.268} / \textbf{1.429}         
        & \textbf{0.144} / \underline{0.646}        
        & \underline{0.189} / \underline{0.388}  
        & \underline{0.535} / \textbf{6.507}                    
        & \textbf{0.342} / \textbf{2.965}           
        & \textbf{0.147} / \textbf{5.889}    \\ \toprule[1pt]
\end{tabular}}
\label{table: metric eval on more datasets.}
\vspace{-1 em}
\end{table*}










\begin{table*}[t]
\centering
\caption{
Comparison with SOTA affine-invariant depth methods on 5 zero-shot transfer benchmarks.
Our model significantly outperforms previous methods and sets new state-of-the-art. Following the benchmark setting, all methods have manually aligned the scale and shift. 
}
\vspace{-1 em}
\setlength{\tabcolsep}{2pt}
\resizebox{0.99\linewidth}{!}{%
\begin{tabular}{ r |ll|ll|ll|ll|ll|ll|l}
\toprule[1pt]
\multirow{2}{*}{Method} & \multirow{2}{*}{Backbone} & \multirow{2}{*}{\#Params} & \multicolumn{2}{c|}{NYUv2} & \multicolumn{2}{c|}{KITTI} & \multicolumn{2}{c|}{DIODE} & \multicolumn{2}{c|}{ScanNet} & \multicolumn{2}{c|}{ETH3D}  & \multicolumn{1}{c}{Rank} \\
&  &   & AbsRel$\downarrow$     & $\delta_{1}$$\uparrow$     & AbsRel$\downarrow$      & $\delta_{1}$$\uparrow$      & AbsRel$\downarrow$      & $\delta_{1}$$\uparrow$      &AbsRel$\downarrow$      & $\delta_{1}$$\uparrow$       &AbsRel$\downarrow$     & $\delta_{1}$$\uparrow$  & \\ \hline
DiverseDepth~\cite{yin2021virtual}& ResNeXt50~\cite{xie2017aggregated}& 25M  
&$0.117$ &$0.875$ 
&$0.190$ &$0.704$ 
&$0.376$ &$0.631$ 
&$0.108$ &$0.882$ 
&$0.228$ &$0.694$  & $7.7$ \\
MiDaS~\cite{Ranftl2020}& ResNeXt101&  88M %
&$0.111$ &$0.885$ 
&$0.236$ &$0.630$ 
&$0.332$ &$0.715$ 
&$0.111$ &$0.886$  
& $0.184$ &$0.752$ & $7.2$ \\
Leres~\cite{leres} & ResNeXt101&  %
&$0.090$  &${0.916}$  
&${0.149}$ &${0.784}$ 
&${0.271}$ &${0.766}$ 
&${0.095}$ &${0.912}$ 
&${0.171}$ &${0.777}$ & $5.4$ \\
Omnidata~\cite{eftekhar2021omnidata} & ViT-base& %
& 0.074 & 0.945 
& 0.149 & 0.835
& 0.339 & 0.742 
& 0.077 & 0.935 
& 0.166 & 0.778 & $4.9$ \\
HDN~\cite{zhang2022hierarchical} & ViT-Large~\cite{dosovitskiy2020an}&  306M  
&$0.069$  &$0.948$  
&$0.115$ &$0.867$ 
&$0.246$ &$0.780$ 
&$0.080$ &$0.939$ 
&$0.121$ &$0.833$ & $3.7$ \\
DPT-large~\cite{ranftl2021vision} & ViT-Large& %
& 0.098 & 0.903 
& 0.10 & 0.901
& \textbf{0.182} & 0.758 
& 0.078 & 0.938 
& 0.078 & 0.946 & $3.8$ \\
\hline
Ours CSTM\_image & ConvNeXt-large~\cite{liu2022convnet}&  198M 
&$\underline{0.058}$  &$\underline{0.963}$  
&$\textbf{0.053}$ &$\underline{0.965}$ 
&$\underline{0.211}$ &$\textbf{0.825}$ 
&$\textbf{0.074}$ &$\textbf{0.942}$ 
&$\textbf{0.064}$ &$\textbf{0.965}$ & $1.3$ \\ 
Ours CSTM\_label & ConvNeXt-large&   
&$\textbf{0.050}$  &$\textbf{0.966}$  
&$\underline{0.058}$ &$\textbf{0.970}$ 
&$0.224$ &$\underline{0.805}$ 
&$\textbf{0.074}$ &$\underline{0.941}$ 
&$\underline{0.066}$ &$\underline{0.964}$ & $1.8$ \\

 \toprule[1pt]
\end{tabular}}

\label{Table: generalization evaluation.}
\vspace{-1.5em}
\end{table*}

\noindent\textbf{Generalization over diverse scenes.}
Affine-invariant depth benchmarks decouple the scale's effect, which aims to evaluate the model's generalization ability to diverse scenes. Recent impact works, such as MiDaS, LeReS, and DPT, achieved promising performance on them. Following them, we test on 5 datasets and manually align the scale and shift to the ground-truth depth before evaluation. Results are reported in Tab.~\ref{Table: generalization evaluation.}. Although our method enforces the network to recover more challenging metric information, our method outperforms them by a large margin on most datasets. 



\begin{figure}[!bth]
\centering
\includegraphics[width=0.5\textwidth]{./files/metrology_in_the_wild.pdf}
\vspace{-2 em}
\caption{\textbf{Reconstruction of in-the-wild scenes.} We collect several Flickr photos, which are captured by various cameras. With photos' metadata, we reconstruct the 3D metric shape and measure structures' sizes. Red and blue marks are ours and ground-truth sizes respectively. }
\label{fig: reconstruction in the wild.}
\vspace{-1em}
\end{figure}


\subsection{Applications Based on Our Method}
In these experiments, we apply the CSTM\_image model to various tasks. 

\noindent\textbf{3D scene reconstruction .}
To demonstrate our work can recover the 3D metric shape in the wild, we first do the quantitative comparison on 9 NYUv2 scenes, which are unseen during training. We predict the per-frame metric depth and then fuse them together with provided camera poses. Results are reported in Tab. \ref{tab: NYUD reconstruction cmp.}. We compare with the video consistent depth prediction method (RCVD~\cite{kopf2021rcvd}), the unsupervised video depth estimation method (SC-DepthV2~\cite{bian2021tpami}), the 3D scene shape recovery method (LeReS~\cite{leres}), affine-invariant depth estimation method (DPT~\cite{ranftl2021vision}), and the multi-view stereo reconstruction method (DPSNet~\cite{im2019dpsnet}). Apart from DPSNet and our method, other methods have to align the scale with the ground truth depth for each frame. Although our method does not aim for the video or multi-view reconstruction problem, our method can achieve promising consistency between frames and reconstruct much more accurate 3D scenes than others on these zero-shot scenes.  From the qualitative comparison in Fig.~\ref{fig: visual nyud reconstruction cmp.}. our reconstructions have much less noise and outliers. 

\noindent\textbf{Dense-SLAM mapping.}
Monocular SLAM is an important robotics application. It only relies on a monocular video input to create the trajectory and dense 3D mapping. Owing to limited photometric and geometric constraints, existing methods face serious scale drift problems in large scenes and cannot recover the metric information. Our robust metric depth estimation method is a strong depth prior to the SLAM system. To demonstrate this benefit,  we naively input our metric depth to the SOTA SLAM system, Droid-SLAM~\cite{teed2021droid}, and evaluate the trajectory on KITTI. We do not do any tuning on the original system. Trajectory comparisons are reported in Tab. \ref{tab: KITTI SLAM.}. As Droid-SLAM can access accurate per-frame metric depth, like an RGB-D SLAM, the translation drift ($t_{rel}$) decreases significantly. Furthermore, with our depths, Droid-SLAM can perform denser and more accurate 3D mapping. An example is shown in Fig.~\ref{Fig: first page fig.} and more cases are shown in the supplementary materials.   

We also test on the ETH3D SLAM benchmarks. Results are reported in Tab.~\ref{Tab: ETH3D SLAM}. Droid with our depths has much better SLAM performance. As the ETH3D scenes are all small-scale indoor scenes, the performance improvement is less than that on KITTI. 

\begin{table}[]
\renewcommand\arraystretch{1.1}
\caption{Comparison with SOTA SLAM methods on KITTI. We input predicted metric depth to the Droid-SLAM~\cite{teed2021droid} (`Droid+Ours'), which outperforms others by a large margin on trajectory accuracy.}
\centering
\resizebox{.98\linewidth}{!}{%
  \centering
  \small 
  \setlength{\tabcolsep}{0.5mm}{\begin{tabular}{@{} l |c|c|c|c|c|c|c@{}}
    \toprule
    \multirow{2}{*}{Method} & Seq 00 & Seq 02 & Seq 05 & Seq 06 & Seq 08 & Seq 09 & Seq 10 \\ \cline{2-8} 
      & \multicolumn{7}{c}{Translational RMS drift ($t_{rel}, \downarrow$) / Rotational RMS drift ($r_{rel}, \downarrow$)} \\ \hline
    GeoNet~\cite{yin2018geonet} & 27.6/5.72 & 42.24/6.14 & 
             20.12/7.67 & 
             9.28/4.34 & 
             18.59/7.85 & 
             23.94/9.81 & 
             20.73/9.1  \\
    VISO2-M~\cite{song2015high}  & 12.66/2.73 & 
             9.47/1.19 & 
             15.1/3.65 & 
             6.8/1.93 & 
             14.82/2.52 & 
             3.69/1.25 & 
             21.01/3.26  \\

    ORB-V2~\cite{murORB2} & 11.43/0.58 & 
             10.34/0.26 &
             9.04/0.26 & 
             14.56/0.26 & 
             11.46/0.28 & 
             9.3/0.26 & 
             2.57/0.32    \\
              
    Droid~\cite{teed2021droid} & 33.9/\textbf{0.29} & 
             34.88/\textbf{0.27} & 
             23.4/0.27 & 
             17.2/0.26 & 
             39.6/0.31 & 
             21.7/0.23 & 
             7/0.25   \\  \hline
             
    Droid+Ours & \textbf{1.44}/0.37 & 
             \textbf{2.64}/0.29 & 
             \textbf{1.44}/\textbf{0.25} & 
             \textbf{0.6}/\textbf{0.2} & 
             \textbf{2.2}/\textbf{0.3} & 
             \textbf{1.63}/\textbf{0.22} & 
             \textbf{2.73}/\textbf{0.23}    \\

    \bottomrule
  \end{tabular}}}
  \label{tab: KITTI SLAM.}
\end{table}

\begin{table}[t]
\caption{
Comparison of VO error on ETH3D benchmark. Droid SLAM system is input with our depth (`Droid + Ours'), and ground-truth depth (`Droid + GT'). The average trajectory error is reported.
}
 \resizebox{\linewidth}{!}{%
\begin{tabular}{l|llllll}
\hline
             & Einstein\_global & Manquin4 & Motion1 & Plantscene3 & sfm\_house\_loop & sfm\_lab\_room2 \\ \hline
& \multicolumn{6}{c}{Average trajectory error ($\downarrow$)}  \\ \hline
Droid        & 4.7                               & 0.88     & 0.83    & 0.78        & 5.64             & 0.55            \\ 
%Droid+LeReS  & 7.8                               & 0.91     & 0.95    & 0.80        & 6.9              & 0.55            \\ \hline
Droid + Ours & 1.5                               & 0.69     & 0.62    & 0.34        & 4.03             & 0.53            \\ 
Droid + GT   & 0.7                               & 0.006    & 0.024   & 0.006       & 0.96             & 0.013           \\ \hline
\end{tabular}}
\label{Tab: ETH3D SLAM}
\vspace{-1 em}
\end{table}

\noindent\textbf{Metrology in the wild.} To show the robustness and accuracy of our recovered metric 3D, we download Flickr photos captured by various cameras and collect coarse camera intrinsic parameters from their metadata. We use our CSTM\_image model to reconstruct their metric shape and measure structures' sizes (marked in red in Fig.~\ref{fig: reconstruction in the wild.}), while the ground-truth sizes are in blue. It shows that our measured sizes are very close to the ground-truth sizes. 






\subsection{Ablation Study}
\noindent\textbf{Ablation on canonical transformation.}
We study the effect of our proposed canonical transformation for the input images (`CSTM\_input') and the canonical transformation for the ground-truth labels (`CSTM\_output'). Results are reported in  Tab. \ref{table: importance of camera model.}. We train the model on sampled mixed data (55K images) and test it on 6 datasets. A naive baseline (`Ours w/o CSTM') is to remove CSTM modules and enforce the same supervision as ours. Without CSTM, the model is unable to converge when training on mixed metric datasets and cannot achieve metric prediction ability on zero-shot datasets. This is why recent mixed-data training methods compromise learning the affine-invariant depth to avoid metric issues. In contrast, our two CSTM methods both can enable the model to achieve the metric prediction ability, and they can achieve comparable performance. Tab. \ref{table:errors cmp on NYUD-V2} also shows comparable performance. Therefore, both adjusting the supervision and the input image appearance during training can solve the metric ambiguity issues. Furthermore, we compare with CamConvs~\cite{facil2019cam}, which encodes the camera model in the decoder with a 4-channel feature. `CamConvs' employ the same training schedule, model, and training data as ours. This method enforces the network to implicitly understand various camera models from the image appearance and then bridges the imaging size to the real-world size. We believe that this method challenges the data diversity and network capacity, thus their performance is worse than ours. 


\begin{table}[]
\caption{Effectiveness of our CSTM. CamConvs~\cite{facil2019cam} directly encodes various camera models in the network, while we perform a simple yet effective transformation to solve the metric ambiguity. Without CSTM, the model cannot achieve transferable metric prediction ability.}
\vspace{-1 em}
\scalebox{0.65}{
\begin{tabular}{l|lll|lll}
\toprule[1pt]
\multirow{2}{*}{Method}        & DDAD & Lyft & DS & NS & KITTI & NYU \\ 
        &\multicolumn{3}{c|}{Test set of train. data (AbsRel$\downarrow$)}     & \multicolumn{3}{c}{Zero-shot test set (AbsRel$\downarrow$)} \\  \hline 
w/o CSTM &$0.530$ &$0.582$  &$0.394$  &$1.00$ & $0.568$      &$0.584$  \\
CamConvs~\cite{facil2019cam}  &$0.295$ &$0.315$  &$0.213$ &$0.423$  &$0.178$      &$0.333$   \\
Ours CSTM\_image &$0.190$ &$0.235$  &$0.182$  &$0.197$ & $0.097$      &$0.210$ \\
Ours CSTM\_label &$0.183$ &$0.221$  &$0.201$  &$0.213$ & $0.081$      &$0.212$  \\
\toprule[1pt]
\end{tabular}}
\label{table: importance of camera model.}
\vspace{-1 em}
\end{table}





\noindent\textbf{Ablation on canonical space.}
We study the effect of the canonical camera here, \textit{i.e.}, the canonical focal length. We train the model on the small sampled dataset and test it on the validation set of training data and testing data. The average AbsRel error is calculated.  We experiment on 3 different focal lengths, \ie, 500, 1000, 1500. Experiments show that $focal=1000$ has slightly better performance than others, see Fig.~\ref{fig: canonical focal length.} for details. Thus we set the canonical focal length to 1000 in our experiments.

\begin{figure}[]
\centering
\includegraphics[width=0.3\textwidth]{./files/precision.pdf}
\caption{\textbf{Effect of different canonical focal lengths.} We experiment on different canonical focal lengths and find that too large or small focal lengths will impact the performance. }
\label{fig: canonical focal length.}
\end{figure}

\noindent\textbf{Effectiveness of 
the random proposal normalization loss.}
To show the effectiveness of our proposed random proposal normalization loss (RPNL), we experiment on the sampled small dataset. Results are shown in Tab.~\ref{table: effectiveness of rpnl.}. We test on the DDAD, Lyft, DrivingStereo (DS), NuScenes (NS), KITTI, and NYUv2.  The `baseline' employs all losses except our RPNL. We compare it with `baseline + RPNL' and `baseline + SSIL~\cite{Ranftl2020}'. We can observe that our proposed random proposal normalization loss can further improve the performance. 
In 
contrast, the scale-shift invariant loss~\cite{Ranftl2020}, which does the normalization on the whole image, can only slightly improve the performance. 
\begin{table}[]
\caption{Effectiveness of random proposal normalization loss. Baseline is supervised by `$L_{\PWN} + L_{\VNL} + L_{silog}$'. SSIL is the scale-shift invariant loss proposed in ~\cite{Ranftl2020}.}
\vspace{-1 em}
\scalebox{0.65}{
\begin{tabular}{l|lll|lll}
\toprule[1pt]
\multirow{2}{*}{Method}        & DDAD & Lyft & DS & NS & KITTI & NYUv2 \\ 
        &\multicolumn{3}{c|}{Test set of train. data (AbsRel$\downarrow$)}     & \multicolumn{3}{c}{Zero-shot test set (AbsRel$\downarrow$)} \\  \hline 
baseline  &$0.204$ &$0.251$  &$0.184$  &$0.207$ &$0.104$      &$0.230$     \\
baseline + SSIL~\cite{Ranftl2020} &$0.197$ &$0.263$  &$0.259$  &$0.206$ & $0.105$      &$0.216$     \\
baseline + RPNL   &$\textbf{0.190}$  &$\textbf{0.235}$  &$\textbf{0.182}$  &$\textbf{0.197}$ &$\textbf{0.097}$      &$\textbf{0.210}$     \\  \toprule[1pt]
\end{tabular}}
\label{table: effectiveness of rpnl.}
\vspace{-2 em}
\end{table}



\section{Conclusion} In this paper, we 
tackle 
the problem of reconstructing the 3D metric scene from a single monocular image. To solve the depth ambiguity in image appearance caused by various focal lengths, we propose a canonical camera space transformation method. With our method, we can easily merge millions of data captured by 10k cameras to train one metric depth model. To improve the robustness, we collected over $8$M data for training. Several zero-shot evaluations show the effectiveness and robustness of our work. We further show the ability to do metrology on randomly collected internet images and dense mapping on large-scale scenes. 

\section*{Acknowledgements}

This work was in part supported by National Key R\&D Program of China (No.\  2022ZD0118700).












\section{Appendix}
%  !TEX root = ../main.tex
~
% \vfill\eject
\section*{Appendix}
\setcounter{equation}{0}
\setcounter{subsection}{0}
\setcounter{section}{0}
\renewcommand{\theequation}{\arabic{equation}}
\renewcommand{\thesubsection}{\arabic{subsection}}
\renewcommand{\thesection}{\Alph{section}}

\section{Single-sideband Amplitude Modulation}\label{append:SSB}
In this section, we give mathematical proof that the baseband perturbation of SSB-AM signals can be recovered by commercial microphones. We initially compare the maximum energy of USB-AM and LSB-AM emitting the same perturbation when sound leakage occurs, and LSB-AM is 87\% of USB-AM. Thus, we adopt the USB-AM in our attacks due to its better inaudibility:
\begin{equation*}
\begin{aligned}
    & \text{USB-AM:}~ S_{USB}(t)=m{cos\omega_{c}t}-\hat{m}{sin\omega_{c}t}+cos\omega_{c}t\\
    & \text{LSB-AM:}~ S_{LSB}(t)=m{cos\omega_{c}t}+\hat{m}{sin\omega_{c}t}+cos\omega_{c}t
\end{aligned}
\label{equ:SSB_formula}
\end{equation*}
where the $\hat{m}$ is the conjugate of $m$. The microphone amplifier's output is below:
$$S_{out} = k_{1}S_{USB}(t) + k_{2}S_{USB}^2(t) + \cdots$$
The $S_{USB}^2(t)$ term has three components: a high-frequency $2\omega_{c}t$ components:
$$(m+1) \hat{m} \sin(2 \omega_c t)+\frac{m^2+2m+1-\hat{m}^2}{2} \cos(2\omega_c t)$$
a direct current (DC) term $\frac{1}{2}$ and an audible component $S_{aud}(t)=\frac{1}{2}(m^2+2m+\hat{m}^2)$.
$S_{USB}(t)$ and the high-frequency component are filtered by the low-pass filter because its frequency is above 25~kHz. The DC component is filtered by the microphone’s capacitor. Thus, the audible component $S_{aud}(t)$ that passes the microphone filtering system can function to ASR.
\vfill\eject

\section{Real-world Scenario}\label{append:attack_scenario}
Figure~\ref{fig:attack_actual} presents our real-world attack scenario.
\begin{figure}[h]
	\centering
	\includegraphics[width=0.48\textwidth]{exp_setup_actual2.pdf}
	\caption{\label{fig:attack_actual}Real-World Attack Scenario.}
	% \vspace{-10pt}
\end{figure}


\section{Algorithm of \alias}\label{append:algo1}
Given that technical workflow for the silence and universal perturbation are overall identical, the major differences are the optimization objective: $y_t/y_b$ and hyper-parameters. Therefore, we demonstrate \alias's representative optimization process of crafting a universal perturbation from scratch in Algorithm.~\ref{algo1}.
% \vfill\eject

\begin{algorithm}[h]
	\caption{Universal \alias Generation}
	\label{algo1}
	\LinesNumbered
	\KwIn{The ASR model with CTC Loss Computation module: $\mathcal{L}$, the maximum epoch: $\text{maxEpoch}$, the desired loss: $objValue$, with a scoring module: $S$, the learning rate: $\eta$, the preset time range: $T$.}
	\KwOut{The universal perturbation $\delta$}
	\textbf{Init} $\delta \gets 0^N$\\
	\For{$1$ to ${ maxEpoch}$}
	{
		${J} \gets 0$\\
            \For{$h_{\theta}\in U_H$, $n\in U_N$}
            {
                $\hat{e} = e^{-a_0 \omega_{c}^{n}d}$\\
                $\overline{\delta} = h_{\theta}\hat{e}\ast \overline{\delta:\hat{\xi}} + n$\\
                \For{$x\in U_x, g\in G, S_{(\cdot)}~{s.t.}~T$}
                {
                    $\Tilde{x}=\beta\cdot g\ast x$\\
                    $\Tilde{x_{\delta}} = clip(\Tilde{x}+\mathcal{S}_{(\overline{\delta})}, [-1,1])$\\
                    $J+=\mathcal{L}(\Tilde{x_{\delta}}, y_t)$
                }
            }
            Compute ${\nabla}_\mathcal{\delta}J$\\
            $\delta \gets \Omega_{Adam}(\delta+\eta\cdot {\nabla}_\mathcal{\delta}J )$\\
            $\delta \gets clip(\delta, [-1,1])$\\
            \If{$J \le objValue$}{break}
        }
	% return $\text{\alias}(\mathcal{A,K})$
	\normalsize
\end{algorithm}
\vfill\eject

\section{Targeted Commands Lists}\label{append:command_list}
Tab.\ref{tab:diff_commands} lists 10 different commands, corresponding to the performance of constructing target command-specific perturbations in experiment \textsection\ref{sec:eval_commands}.
\begin{table}[h]
	\centering
        \normalsize
		\caption{Attack with Different Targeted Commands}
		\renewcommand\arraystretch{0.9}
		\renewcommand\tabcolsep{1.5pt}
		\begin{threeparttable}
			\begin{tabular}{l|c|c}
                \toprule
                \textbf{Target Command} & \textbf{~~~~SR~~~~} & \textbf{~~CER~~} \\
				\midrule
                    ``Start recording''  & 100\% & 0\% \\ \midrule
				``Set a timer''  & 100\%  & 0\% \\ \midrule
				``Open the door''  & 100\% & 0\% \\ \midrule
				``Take the picture''  & 100\% & 0\% \\ \midrule
				``Call nine one one (911)''  & 100\% & 0\% \\ \midrule
				``Cancel my morning alarm''  & 100\% & 0\% \\ \midrule
                    ``Turn on airplane mode'' & 94.39\% & 0.28\% \\ \midrule
                    ``Open my photo album''  & 95.03\% & 0.50\% \\ \midrule
				``What is going on Twitter?''  & 100\% & 0\% \\ \midrule
                    ``Mute volume and turn off the WiFi'' & 92.82\% & 0.21\% \\
				\bottomrule
			\end{tabular}
		% \vspace{-10pt}
		\end{threeparttable}
		\label{tab:diff_commands}
\end{table}
% \vfill\eject


\section{Different Attack Angles}\label{append:eval_angles}
In this experiment, we keep the recording device's bottom microphone spatially within the ultrasound beam's coverage and set the attack distance to 2.5m. We rotate the recording device from 0\degree to 180\degree at 15\degree intervals, among which 90\degree means the ultrasound directly points to the bottom microphone. Under each angle, we play 40 benign commands and emit the universal IAP.
Eventually, we collected 520 mixed audio signals from 13 angles. As shown in Fig.~\ref{fig:attack_angle}, although ultrasound is highly directional, we find that there is no significant difference with 100\% success rate among different angles within 15\degree$\sim$150\degree. As the deployed location of bottom microphones varies with different phones, therefore attack performance is not symmetrical with angles (i.e., 79\% at 0\degree and 49\% at 180\degree). Overall, as most voice-interface devices nowadays are equipped with omnidirectional microphones, \alias can be effective as long as the beam can cover the bottom microphone. 
\begin{figure}[h]
    \centering
    \includegraphics[width=0.45\textwidth]{attack_angle.pdf}
    % \vspace{-10pt}
    \caption{\alias's performance at different angles.}
    \label{fig:attack_angle}
\end{figure}

% \vfill\eject

\section{Different Speech \& Perturbation Loudness}\label{append:loudness_compare}
Fig.~\ref{fig:loudness_compare} shows the success rate and CER of our experiments on the relative energies between the attack perturbation and speech.
\begin{figure}[h]
	% \vspace{-7pt}
	\centering  %居中
		\subfigure[Success Rate (\%)]{   %第一张子图
		\begin{minipage}[t]{0.22\textwidth}
			\centering
                % \hspace{-0.25\linewidth}
			\includegraphics[width=1.1\textwidth]{snr_SR.pdf} % height=0.765\linewidth
		\end{minipage}
		}
	\subfigure[CER (\%)]{   %第一张子图
	\begin{minipage}[t]{0.22\textwidth}
		\centering
            % \hspace{-0.25\linewidth}
		\includegraphics[width=1.1\textwidth]{snr_CER.pdf} 
	\end{minipage}
	}
	% \vspace{-10pt}
	\caption{The performance of loudness relationship between user speech and perturbation.}    %大图名称
	\label{fig:loudness_compare}    %图片引用标记
	% \vspace{-5pt}
\end{figure}
% \vfill\eject

\section{User Testing}\label{append:user_test}
In this section, we elaborate on the \textit{Man-in-the-middle} attack strategy, whose effect is akin to experiencing network congestion when users use the ASR service, resulting in slower responses. Prolonged latency can make users feel uncomfortable while using the service. 
To assess user awareness under such delays, we design 10 scenarios, each consisting of an audio clip that simulates a user issuing a command to the ASR system with random delays (1-5 seconds) before the voice assistant executes the command. 
We collected test results from 140 college students of different majors. As shown in Figure~\ref{fig:user_latency}, when the delay time is less than 2.7 seconds (the junction point of two distribution curves), more users find the ASR service comforting than uncomfortable. 
The participants are also asked to fill in what they think the cause is if they experience an uncomfortable delay when using the ASR service.
Only 11 out of 140 participants suspect an attack, while almost all others attribute the delay to network latency/congestion or device stuck, suggesting that this strategy poses a hidden attack.
We believe that users' suspicion may also be related to their disciplinary background, e.g., users with knowledge of cybersecurity are more likely to consider the possibility of an attack.

\begin{figure}[h]
    \centering
    \includegraphics[width=0.45\textwidth]{latency.pdf}
    \caption{The probability distribution of users' awareness during a \textit{man-in-the-middle} attack under different delay conditions (similar to network latency). ``Comfortable'': the situation where users find the ASR service is normal and are not aware of the attack; ``Uncomfortable'': the delay may cause them to feel uncomfortable or unusual.}
    \label{fig:user_latency}
\end{figure}






{\small
\bibliographystyle{./ICCV2023/ieee_fullname}
\bibliography{./ICCV2023/egbib}
}

\end{document}