%  !TEX root = ../main.tex

\begin{abstract}
Automatic speech recognition (ASR) systems have been shown to be vulnerable to adversarial examples (AEs). Recent success all assumes that users will not notice or disrupt the attack process despite the existence of music/noise-like sounds and spontaneous responses from voice assistants. Nonetheless, in practical user-present scenarios, user awareness may nullify existing attack attempts that launch unexpected sounds or ASR usage. 
In this paper, we seek to bridge the gap in existing research and extend the attack to user-present scenarios. We propose \alias, an inaudible adversarial perturbation (IAP) attack via ultrasound delivery that can manipulate ASRs as a user speaks.
The inherent differences between audible sounds and ultrasounds make IAP delivery face unprecedented challenges such as distortion, noise, and instability. In this regard, we design a novel ultrasonic transformation model to enhance the crafted perturbation to be physically effective and even survive long-distance delivery.
We further enable \alias's robustness by adopting a series of augmentation on user and real-world variations during the generation process.
In this way, \alias features an effective real-time manipulation of the ASR output from different distances and under any speech of users, with an \textit{alter-and-mute} strategy that suppresses the impact of user disruption.
Our extensive experiments in both digital and physical worlds verify \alias's effectiveness under various configurations, robustness against six kinds of defenses, and universality in a targeted manner. We also show that \alias can be delivered with a portable attack device and even everyday-life loudspeakers.
\end{abstract}

