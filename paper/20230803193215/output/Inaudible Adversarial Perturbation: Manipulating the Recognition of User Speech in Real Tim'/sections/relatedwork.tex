%  !TEX root = ../main.tex

\section{Related Work}
\textbf{Custom Adversarial Examples \& Inaudible Attacks.}
The initial AE attacks construct a custom (i.e., non-universal) perturbation for a specific audio clip, whereas the same perturbation cannot compromise other audio. Signal-level transformations~\cite{vaidya2015cocaine,carlini2016hidden,abdullah2019ndss}, such as modifying MFCC, are unintelligible to human beings but can be recognized by the ASR model. As this class of attacks resembles obvious noises, they can easily alert users. Thus, inaudible attacks~\cite{zhang2017dolphinattack,roy2018inaudible,yan2020surfingattack} have been proposed, which exploit carrier signals outside the audible frequencies of human beings (e.g., 40~kHz) to inject attacks into ASR systems utilizing the nonlinearity vulnerability of microphone circuits, yet entirely unheard by victims. However, compared with audible playback speech samples, such attacks usually suffer from signal distortion and low SNR due to their dependence on various convert channels, e.g., ultrasound~\cite{ji2022capspeaker}, laser~\cite{sugawara2020light}, or electricity~\cite{wang2022ghosttalk} signals, and the hardware imperfections these channels introduce.
There is also a major branch of the research community that leverages the vulnerability of ASR models by adding slightly audible perturbations on the benign audio based on $\epsilon$-constraint~\cite{carlini2018audio,taori2019targeted} and psychoacoustic hiding~\cite{schonherr2018adversarial,qin2019imperceptible}, to make the AEs sound benign but fool the ASR's transcription. It is worth noting that non-universal AEs lose effectiveness for streaming speech input and unpredictable user commands, as they rely on perfect temporal alignment. Constructing multiple AEs for altering different commands as an adversary-desired instruction is also impractical. 

\textbf{Universal Adversarial Examples.}
Recent studies propose universal AEs that can apply to tamper with multiple speech content as an adversary-desired command. Existing untargeted universal AE attacks adopt iterative greedy algorithms~\cite{moosavi2017universal} can cause arbitrary speech to mis-classification~\cite{vadillo2019universal} or false transcription~\cite{neekhara2019universal}. In contrast, targeted universal AE attack is very challenging in speech recognition tasks because ASR models are context-dependent, and a certain minor perturbation superimposed even at different positions of a given benign audio, the whole sentence may yield various transcription results. This is distinct from the prior successful targeted universal AE attack in the text-independent speaker recognition~\cite{deng2022fencesitter,li2023tuner} and the universal adversarial patch attack in position-insensitive CNN-based image classification tasks~\cite{brown2017adversarial}. Moreover, given that the victim user can easily notice the audible-band perturbation, AdvPulse~\cite{li2020advpulse} disguises short pulses in the environment sounds to be less perceptible. However, they only apply to a context-insensitive CNN-based audio command classification model to be universal. 
\blue{
To overcome the mainstream RNN-based ASR context-dependent issues, a partial match strategy is proposed by SpecPatch~\cite{guo2022specpatch}, which also employs audible noise-like short pulses (0.5s) to alter multiple short user commands into the targeted instruction against the mainstream DeepSpeech ASR model. However, such an attack will not work in relatively long commands ($\ge$ 4 words) and can be noticed despite following L2-imperceptibility constraints.}

\blue{
Overall, due to the fundamental differences between audible and ultrasonic channels, \alias differs from prior works that encountered challenges related to \textbf{\textit{user auditory}} and \textbf{\textit{user disruption}}. 
In addition to the four representative merits over existing AEs listed in Tab.~\ref{tab:compare}, \alias offers several additional benefits: (1) the optimization process is no longer subject to audibility constraints such as tiny $\epsilon$, psychoacoustics, $L_p$-norm, nor does it need to limit the signal form as short pulses to reduce the possibility of being perceived. (2) \alias's broad optimization space further allows for fewer iterations while maintaining a high degree of universality. Combining these two advantages, \alias enables real-time manipulation of arbitrary user commands and long speech sentences in an \textit{alter-and-mute} fashion, as never before. (3) Unlike audible-band AEs that are easily compromised by interference due to their subtle perturbations, \alias demonstrates robustness and remains effective even when faced with various audio pre-processing defenses.
Notably, our initial modeling of ultrasound transformation precisely characterizes the ultrasound channel and justifies it as a promising carrier for IAP delivery. We believe that this modeling effort lays the groundwork for generating inaudible AEs and may inspire future works.
}
\color{black}
