\subsection{Adjustments of Navigation State for Guidance and Control}

The philosophy used in the design of the navigation filter was to provide the best estimate of the vehicle state and breadcrumbs states in the takeoff frame.  Between the navigation filter and the guidance and control, a set of offsets and filters were added to improve overall system performance.

\subsection{Filter of Step Corrections in State}

Within the Kalman Filter framework, the vehicle state propagates smoothly when only integrating angular rate and acceleration measurements from the IMU.  When other other measurements are received however, there are step changes in the navigation estimate.  It was found that these steps were negatively interacting with the control system and reducing vehicle stability.  Merely adding a low pass filter to the navigation state is not desirable since it would increase phase lag in the system by filtering the entire navigation signal including high rate IMU data.  Thus a filter was added that only filtered the state corrections from the correction step of the filter while allowing high frequency content from the IMU to pass through.  The position estimate sent to the controller, $\hat{r}_{ctrl}$ and $\hat{v}_{ctrl}$, are calculated from raw navigation estimates, $\hat{r}_{navl}$ and $\hat{v}_{nav}$ by:

\begin{equation}
    {\begin{array}{c}
   \hat{r}_{ctrl}(t_k) = \hat{r}_{navl}(t_k) - z(t_k) \\ \\
   z(t_k) = \alpha\left(z(t_{k-1}) + \delta r(t_k)\right) \\
   \end{array}}
\end{equation}

\noindent where $z$ is a running weighted sum of the position corrections in the filter, $\delta r(t_k)$ is the change to the position estimate at time $t_k$ resulting from any non-IMU measurement, and $\alpha$ is a positive constant slightly less than one.  The velocity estimate is filtered in the same fashion.  The constant $alpha$ was chosen so that each measurement is gradually fed into the position and velocity estimates seen by the controller over several seconds.


\subsection{Correcting for Historic Breadcrumb Location Drift}

During most flights, a path is defined for the Dragonfly vehicle to fly a trajectory that covers a set of waypoints from a previous flight.  This trajectory is defined prior to each flight based on the estimated breadcrumb locations in the database from the previous flight.  However, the current filter implementation allows both the estimate of the vehicle position and the estimation of the currently tracked breadcrumb to be corrected in the filter.  The update to the breadcrumb location must be accounted for in the vehicle target trajectories.  While it could be possible to define a guidance law to explicitly track breadcrumb locations or to recalculate the trajectory whenever the breadcrumb state in the filter is updated, a simpler approach was identified.  A correction term is calculated between the navigation filter by taking the difference in the estimate of the historic breadcrumb in the filter and the position estimate for that same breadcrumb in the database.  The breadcrumb position estimates in the database were those used to define the target trajectory. This offset is provided to the guidance law to ensure that the vehicle properly flies over historic breadcrumbs.  A low pass filter is also applied to this correction to prevent jumps in the estimate from impacting low level control.  